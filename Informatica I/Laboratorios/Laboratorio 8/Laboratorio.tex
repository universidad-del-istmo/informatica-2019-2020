%%%%%%%%%%%%%%%%%%%%%%%%%%%%%%%%%%%%%%%%%
% Programming/Coding Assignment
% LaTeX Template
%
% This template has been downloaded from:
% http://www.latextemplates.com
%
% Original author:
% Ted Pavlic (http://www.tedpavlic.com)
%
% Note:
% The \lipsum[#] commands throughout this template generate dummy text
% to fill the template out. These commands should all be removed when 
% writing assignment content.
%
% This template uses a Perl script as an example snippet of code, most other
% languages are also usable. Configure them in the "CODE INCLUSION 
% CONFIGURATION" section.
%
%%%%%%%%%%%%%%%%%%%%%%%%%%%%%%%%%%%%%%%%%

%----------------------------------------------------------------------------------------
%	PACKAGES AND OTHER DOCUMENT CONFIGURATIONS
%----------------------------------------------------------------------------------------

\documentclass{article}

\usepackage{fancyhdr} % Required for custom headers
\usepackage{lastpage} % Required to determine the last page for the footer
\usepackage{extramarks} % Required for headers and footers
\usepackage[usenames,dvipsnames]{color} % Required for custom colors
\usepackage{graphicx} % Required to insert images
\usepackage{listings} % Required for insertion of code
\usepackage{courier} % Required for the courier font
\usepackage{multirow}
\usepackage{hyperref}
\usepackage{amsmath}
\usepackage{amssymb}

% Margins
\topmargin=-0.45in
\evensidemargin=0in
\oddsidemargin=0in
\textwidth=6.5in
\textheight=9.0in
\headsep=0.25in

\linespread{1.1} % Line spacing

%----------------------------------------------------------------------------------------
%	CODE INCLUSION CONFIGURATION
%----------------------------------------------------------------------------------------

\definecolor{MyDarkGreen}{rgb}{0.0,0.4,0.0} % This is the color used for comments
\lstloadlanguages{c} % Load Perl syntax for listings, for a list of other languages supported see: ftp://ftp.tex.ac.uk/tex-archive/macros/latex/contrib/listings/listings.pdf
\lstset{language=[sharp]c, % Use Perl in this example
        frame=single, % Single frame around code
        basicstyle=\small\ttfamily, % Use small true type font
        keywordstyle=[1]\color{Blue}\bf, % Perl functions bold and blue
        keywordstyle=[2]\color{Purple}, % Perl function arguments purple
        keywordstyle=[3]\color{Blue}\underbar, % Custom functions underlined and blue
        identifierstyle=, % Nothing special about identifiers                                         
        commentstyle=\usefont{T1}{pcr}{m}{sl}\color{MyDarkGreen}\small, % Comments small dark green courier font
        stringstyle=\color{Purple}, % Strings are purple
        showstringspaces=false, % Don't put marks in string spaces
        tabsize=5, % 5 spaces per tab
        %
        % Put standard Perl functions not included in the default language here
        morekeywords={rand},
        %
        % Put Perl function parameters here
        morekeywords=[2]{on, off, interp},
        %
        % Put user defined functions here
        morekeywords=[3]{test},
       	%
        morecomment=[l][\color{Blue}]{...}, % Line continuation (...) like blue comment
        numbers=left, % Line numbers on left
        firstnumber=1, % Line numbers start with line 1
        numberstyle=\tiny\color{Blue}, % Line numbers are blue and small
        stepnumber=5 % Line numbers go in steps of 5
}

\newcommand{\horrule}[1]{\rule{\linewidth}{#1}}

% Creates a new command to include a perl script, the first parameter is the filename of the script (without .pl), the second parameter is the caption
\newcommand{\perlscript}[2]{
\begin{itemize}
\item[]\lstinputlisting[caption=#2,label=#1]{#1.cs}
\end{itemize}
}

\begin{document}

\begin{tabular}{l l}
\multirow{5}{*}{\includegraphics[width=2cm]{../../recursos/logo.png}}
 & Universidad del Istmo de Guatemala \\
 & Facultad de Ingenieria \\
 & Ing. en Sistemas \\
 & Informatica 1 \\
 & Prof. Ernesto Rodriguez - \href{mailto:erodriguez@unis.edu.gt}{erodriguez@unis.edu.gt} \\
\end{tabular}
\\\\\\

\begin{center}
        \horrule{0.5pt}
        \huge{Hoja de trabajo \#8} \\
        \large{Fecha de entrega: 24 de Septiembre, 2019 - 11:59pm} \\
        \horrule{1pt}
\end{center}

\emph{Instrucciones: Resolver cada uno de los ejercicios siguiendo sus respectivas
instrucciones. El trabajo debe ser entregado a traves de Github, en su repositorio del curso, colocado en una
carpeta llamada "Laboratorio 8". Al menos que la pregunta indique diferente, todas las
respuestas a preguntas escritas deben presentarse en un documento formato pdf, el cual
haya sido generado mediante Latex. }\\\\

{\bf Nota: } Para esta tarea, debe tener instalado ``Elm'' en su computadora. Puede obtener
el lenguaje ``Elm'' en: https://guide.elm-lang.org/install.html

\section*{Ejercicio \#1 (20\%)}

Modifique el \emph{arbol binario} definido en el laboratorio \#7 de tal
forma que pueda contener un valor de cualquier tipo; en vez de solo
poder tener $\mathbf{Int}$. A lo largo de este trabajo se utilizaran
letras minusculas antecedidas de un apostrofe (ej. $\mathtt{'t}$) para indicar que se
esta trabajando con un tipo generico. Para indicar que un tipo es \emph{generico},
se colocara el numbre del tipo seguido de una variable de tipo (ej. $\mathbf{Arbol}\ \mathtt{`t}$)

\section*{Ejercicio \#2 (20\%)}

Defina una funci\'on llamada $\mathtt{map}\ :\ (\mathtt{'t}\rightarrow\mathtt{'u})\rightarrow
(\mathbf{Arbol}\ \mathtt{'t}) \rightarrow (\mathbf{Arbol}\ \mathtt{'u})$. Esta funci\'on toma un arbol y aplica la funci\'on
pasada como primer parametro a cada elemento del arbol.

\section*{Ejercicio \#3 (25\%)}

Defina una funci\'on llamada $\mathtt{filtrar}\ :\ (\mathtt{'t}\rightarrow\mathbf{Bool})\rightarrow (\mathbf{Arbol}\ \mathtt{'t})
\rightarrow (\mathbf{List}\ \mathtt{'t})$. Esta funci\'on recibe una condicion y busca todos los elementos
del arbol que statisfagan esa condici\'on.

\section*{Ejercicio \#4 (25\%)}

Defina una funci\'on llamada $\mathtt{foldTree}\ :\ (\mathtt{'s}\rightarrow\mathtt{'t}
\rightarrow\mathtt{'t}\rightarrow\mathtt{'s})\rightarrow\mathtt{'s}\rightarrow(\mathbf{Arbol}\ \mathtt{'t})\rightarrow
\mathtt{'s}$.
Esta funci\'on recorre todos los elementos del arbol. Si el elemento se encuentra vacio,
simplemente retorna el valor dado como segundo parametro, de lo contrario llama la funci\'on
$\mathtt{foldTree}$ recursivamente en cada uno de los arboles y luego le aplica a los
dos resultados y al valor del elemento del arbol.

\section*{Ejercicio \#5 (10\%)}

Defina nuevamente la funci\'on $\mathtt{filtrar}$ utilizando la funci\'on $\mathtt{foldTree}$ para
hacer la recursi\'on. Llame a esta funci\'on $\mathtt{filtrarFold}$

\end{document}