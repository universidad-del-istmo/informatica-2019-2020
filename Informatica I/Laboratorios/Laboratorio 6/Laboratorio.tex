%%%%%%%%%%%%%%%%%%%%%%%%%%%%%%%%%%%%%%%%%
% Programming/Coding Assignment
% LaTeX Template
%
% This template has been downloaded from:
% http://www.latextemplates.com
%
% Original author:
% Ted Pavlic (http://www.tedpavlic.com)
%
% Note:
% The \lipsum[#] commands throughout this template generate dummy text
% to fill the template out. These commands should all be removed when 
% writing assignment content.
%
% This template uses a Perl script as an example snippet of code, most other
% languages are also usable. Configure them in the "CODE INCLUSION 
% CONFIGURATION" section.
%
%%%%%%%%%%%%%%%%%%%%%%%%%%%%%%%%%%%%%%%%%

%----------------------------------------------------------------------------------------
%	PACKAGES AND OTHER DOCUMENT CONFIGURATIONS
%----------------------------------------------------------------------------------------

\documentclass{article}

\usepackage{fancyhdr} % Required for custom headers
\usepackage{lastpage} % Required to determine the last page for the footer
\usepackage{extramarks} % Required for headers and footers
\usepackage[usenames,dvipsnames]{color} % Required for custom colors
\usepackage{graphicx} % Required to insert images
\usepackage{listings} % Required for insertion of code
\usepackage{courier} % Required for the courier font
\usepackage{multirow}
\usepackage{hyperref}
\usepackage{amsmath}
\usepackage{amssymb}

% Margins
\topmargin=-0.45in
\evensidemargin=0in
\oddsidemargin=0in
\textwidth=6.5in
\textheight=9.0in
\headsep=0.25in

\linespread{1.1} % Line spacing

%----------------------------------------------------------------------------------------
%	CODE INCLUSION CONFIGURATION
%----------------------------------------------------------------------------------------

\definecolor{MyDarkGreen}{rgb}{0.0,0.4,0.0} % This is the color used for comments
\lstloadlanguages{c} % Load Perl syntax for listings, for a list of other languages supported see: ftp://ftp.tex.ac.uk/tex-archive/macros/latex/contrib/listings/listings.pdf
\lstset{language=[sharp]c, % Use Perl in this example
        frame=single, % Single frame around code
        basicstyle=\small\ttfamily, % Use small true type font
        keywordstyle=[1]\color{Blue}\bf, % Perl functions bold and blue
        keywordstyle=[2]\color{Purple}, % Perl function arguments purple
        keywordstyle=[3]\color{Blue}\underbar, % Custom functions underlined and blue
        identifierstyle=, % Nothing special about identifiers                                         
        commentstyle=\usefont{T1}{pcr}{m}{sl}\color{MyDarkGreen}\small, % Comments small dark green courier font
        stringstyle=\color{Purple}, % Strings are purple
        showstringspaces=false, % Don't put marks in string spaces
        tabsize=5, % 5 spaces per tab
        %
        % Put standard Perl functions not included in the default language here
        morekeywords={rand},
        %
        % Put Perl function parameters here
        morekeywords=[2]{on, off, interp},
        %
        % Put user defined functions here
        morekeywords=[3]{test},
       	%
        morecomment=[l][\color{Blue}]{...}, % Line continuation (...) like blue comment
        numbers=left, % Line numbers on left
        firstnumber=1, % Line numbers start with line 1
        numberstyle=\tiny\color{Blue}, % Line numbers are blue and small
        stepnumber=5 % Line numbers go in steps of 5
}

\newcommand{\horrule}[1]{\rule{\linewidth}{#1}}

% Creates a new command to include a perl script, the first parameter is the filename of the script (without .pl), the second parameter is the caption
\newcommand{\perlscript}[2]{
\begin{itemize}
\item[]\lstinputlisting[caption=#2,label=#1]{#1.cs}
\end{itemize}
}

\begin{document}

\begin{tabular}{l l}
\multirow{5}{*}{\includegraphics[width=2cm]{../../recursos/logo.png}}
 & Universidad del Istmo de Guatemala \\
 & Facultad de Ingenieria \\
 & Ing. en Sistemas \\
 & Informatica 1 \\
 & Prof. Ernesto Rodriguez - \href{mailto:erodriguez@unis.edu.gt}{erodriguez@unis.edu.gt} \\
\end{tabular}
\\\\\\

\begin{center}
        \horrule{0.5pt}
        \huge{Hoja de trabajo \#6} \\
        \large{Fecha de entrega: 10 de Septiembre, 2019 - 11:59pm} \\
        \horrule{1pt}
\end{center}

\emph{Instrucciones: Resolver cada uno de los ejercicios siguiendo sus respectivas
instrucciones. El trabajo debe ser entregado a traves de Github, en su repositorio del curso, colocado en una
carpeta llamada "Laboratorio 6". Al menos que la pregunta indique diferente, todas las
respuestas a preguntas escritas deben presentarse en un documento formato pdf, el cual
haya sido generado mediante Latex. }\\\\

{\bf Nota: } Para esta tarea, debe tener instalado ``Elm'' en su computadora. Puede obtener
el lenguaje ``Elm'' en: https://guide.elm-lang.org/install.html

\section*{Ejercicio \#1 (25\%)}

Escriba una funcion llamada $\mathtt{iFilter\ :\ \mathbb{Z}\ \rightarrow\ \mathtt{List}\ \mathbb{Z}\ \rightarrow\ \mathtt{List}\ \mathbb{Z}}$. Esta
funci\'on debe tomar un entero y una lista de enteros y retornar una nueva lista que tenga todos los valores de la lista
original excepto los valores divisibles dentro del numero dado.

\section*{Ejercicio \#2 (25\%)}

Escriba una funci\'on llamada $\mathtt{filter}\ :\ (\mathbb{Z}\rightarrow\mathbb{B})
\ \rightarrow\ \mathtt{List}\ \mathbb{Z}\ \rightarrow\ \mathtt{List}\ \mathbb{Z}$. Esta funci\'on toma
como primer parametro otra funci\'on que va de numeros enteros a booleanos. Esta
funci\'on debe ser llamada con todos los elementos de la lista. La lista retornada
por esta funci\'on debe tener solamente los elementos para los cuales la funci\'on
que se paso como parametro retorna $\mathbb{True}$.

\section*{Ejercicio \#3 (25\%)}

Escriba una funci\'on llamada $\mathtt{iZipWith}\ :\ \mathtt{List}\ \mathbb{Z}\ \rightarrow\ \mathtt{List}\ \mathbb{Z}
\rightarrow\ \mathtt{List}\ \mathbb{Z}$. Esta funci\'on toma dos listas de numeros y retorna
una lista nueva en donde cada elemento es la suma de los elementos de las listas
originales que aparecen en la misma posici\'on. Por ejemplo: $\mathtt{iZipWith}\ [1,2,3]\ 
[4,5,6]$ produce la lista $[5,7,9]$.

\section*{Ejercicio \#4 (25\%)}

Escriba una funci\'on llamada $\mathtt{zipWith}\ :\ (\mathbb{Z}\rightarrow\mathbb{Z}\rightarrow\mathbb{Z})\ 
\rightarrow\ \mathtt{List}\ \mathbb{Z}\rightarrow\ \mathtt{List}\ \mathbb{Z}\ \rightarrow\ \mathtt{List}\ \mathbb{Z}$. Esta
funci\'on es similar a $\mathtt{iZipWith}$ ya que combina los elementos de las dos
listas que aparecen en la misma posici\'on. Sin embargo en vez de sumar los elementos,
toma una funci\'on que toma dos enteros y produce un entero de tal forma que los
elementos de la lista resultante son los elementos producios por la funcion dada
como primer parametro.

\end{document}