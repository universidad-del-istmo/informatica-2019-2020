%%%%%%%%%%%%%%%%%%%%%%%%%%%%%%%%%%%%%%%%%
% Programming/Coding Assignment
% LaTeX Template
%
% This template has been downloaded from:
% http://www.latextemplates.com
%
% Original author:
% Ted Pavlic (http://www.tedpavlic.com)
%
% Note:
% The \lipsum[#] commands throughout this template generate dummy text
% to fill the template out. These commands should all be removed when 
% writing assignment content.
%
% This template uses a Perl script as an example snippet of code, most other
% languages are also usable. Configure them in the "CODE INCLUSION 
% CONFIGURATION" section.
%
%%%%%%%%%%%%%%%%%%%%%%%%%%%%%%%%%%%%%%%%%

%----------------------------------------------------------------------------------------
%	PACKAGES AND OTHER DOCUMENT CONFIGURATIONS
%----------------------------------------------------------------------------------------

\documentclass{article}

\usepackage{fancyhdr} % Required for custom headers
\usepackage{lastpage} % Required to determine the last page for the footer
\usepackage{extramarks} % Required for headers and footers
\usepackage[usenames,dvipsnames]{color} % Required for custom colors
\usepackage{graphicx} % Required to insert images
\usepackage{listings} % Required for insertion of code
\usepackage{courier} % Required for the courier font
\usepackage{multirow}
\usepackage{hyperref}


% Margins
\topmargin=-0.45in
\evensidemargin=0in
\oddsidemargin=0in
\textwidth=6.5in
\textheight=9.0in
\headsep=0.25in

\linespread{1.1} % Line spacing

%----------------------------------------------------------------------------------------
%	CODE INCLUSION CONFIGURATION
%----------------------------------------------------------------------------------------

\definecolor{MyDarkGreen}{rgb}{0.0,0.4,0.0} % This is the color used for comments
\lstloadlanguages{c} % Load Perl syntax for listings, for a list of other languages supported see: ftp://ftp.tex.ac.uk/tex-archive/macros/latex/contrib/listings/listings.pdf
\lstset{language=[sharp]c, % Use Perl in this example
        frame=single, % Single frame around code
        basicstyle=\small\ttfamily, % Use small true type font
        keywordstyle=[1]\color{Blue}\bf, % Perl functions bold and blue
        keywordstyle=[2]\color{Purple}, % Perl function arguments purple
        keywordstyle=[3]\color{Blue}\underbar, % Custom functions underlined and blue
        identifierstyle=, % Nothing special about identifiers                                         
        commentstyle=\usefont{T1}{pcr}{m}{sl}\color{MyDarkGreen}\small, % Comments small dark green courier font
        stringstyle=\color{Purple}, % Strings are purple
        showstringspaces=false, % Don't put marks in string spaces
        tabsize=5, % 5 spaces per tab
        %
        % Put standard Perl functions not included in the default language here
        morekeywords={rand},
        %
        % Put Perl function parameters here
        morekeywords=[2]{on, off, interp},
        %
        % Put user defined functions here
        morekeywords=[3]{test},
       	%
        morecomment=[l][\color{Blue}]{...}, % Line continuation (...) like blue comment
        numbers=left, % Line numbers on left
        firstnumber=1, % Line numbers start with line 1
        numberstyle=\tiny\color{Blue}, % Line numbers are blue and small
        stepnumber=5 % Line numbers go in steps of 5
}

\newcommand{\horrule}[1]{\rule{\linewidth}{#1}}

\newcommand\doubleplus{\ensuremath{\mathbin{+\mkern-10mu+}}}

% Creates a new command to include a perl script, the first parameter is the filename of the script (without .pl), the second parameter is the caption
\newcommand{\perlscript}[2]{
\begin{itemize}
\item[]\lstinputlisting[caption=#2,label=#1]{#1.cs}
\end{itemize}
}

\begin{document}

\begin{tabular}{l l}
\multirow{5}{*}{\includegraphics[width=2cm]{../../recursos/logo.png}} & Universidad del Istmo de Guatemala \\
 & Facultad de Ingenieria \\
 & Ing. en Sistemas \\
 & Informatica II \\
 & Prof. Ernesto Rodriguez - \href{mailto:erodriguez@unis.edu.gt}{erodriguez@unis.edu.gt} \\
\end{tabular}
\\\\\\

\begin{center}
        \horrule{0.5pt}
        \huge{Examen Parcial \#1} \\
        \large{Fecha de entrega: 17 de Febrero, 2019 - 11:59pm} \\
        \horrule{1pt}
\end{center}

\emph{Instrucciones: Resolver cada uno de los ejercicios siguiendo sus respectivas
instrucciones. El trabajo debe ser entregado a traves de Blackboard. Debbe ser resuelto de
forma {\bf individual} y el plagio sera sancionado de forma severa. Si tiene dudas o necesita
apoyo para resolver el examen, puede escribirme un correo.}

\section*{Introducci\'on}

Es possible \emph{codificar} datos dentro de la representaci\'on binaria de un numero. A esto se le conoce
como \emph{codificaci\'on binaria} y suele ser mucho m\'as eficiente y compacata que la codificaci\'on hecha
en texto. Por ejemplo, la luz tiene 3 colores primaris: rojo, verde y azul. Cualquier color se puede crear
mezclando estos tres colores, adicionalmente, se define algo llamado \emph{transparencia} que indica con que
intensidad se expresa el color final. A esta codificaci\'on se le conoce como \emph{rgba}. Agunos ejemplos:
\begin{itemize}
    \item{$\langle 255,\ 0,\ 0,\ 255\rangle$ representa el color rojo fuerte.}
    \item{$\langle 0,\ 255,\ 0,\ 255\rangle$ representa el color verde fuerte.}
    \item{$\langle 255,\ 255,\ 0,\ 255\rangle$ representa el color amarillo fuerte ya que es la combinacion de un 50\% rojo y 50\% verde.}
    \item{$\langle 255,\ 128,\ 0,\ 128\rangle$ representa un color naranja poco intenso (debido a que la transparencia es la mitad).}
\end{itemize}
Dado que cada uno de los parametros que forman un color se encuentra entre $0$ y $255$, se pueden representar
con un $\mathbf{unsigned\ short}$. Dado que cada short tiene 8 bits de longitud (o un byte), un color se puede
codificar en un $\mathbf{unsigned\ int}$ en donde:
\begin{itemize}
    \item{Los bits 0 al 7 representan el numero correspondiente al color rojo.}
    \item{Los bits 8 al 15 representan el numero correspondiente al color verde.}
    \item{Los bits 16 al 23 representan el numero correspondiente al color azul. }
    \item{Los bits 24 al 32 representan el numero correspondiente a la transparencia. }
\end{itemize}
Por ejemplo, el color $\langle 255,\ 128,\ 0,\ 128\rangle$ tendria el valor binaro de $10000000000000001000000011111111$.
\\\\Usted ha sido contratado por una empresa que tiene una flotilla de caminoes que se mueven \emph{exclusivamente en
la ciudad de Guatemala}.
Cada camion esta equipado con una \emph{unidad de GPS y monitoreo} que le proporciona los siguientes datos:
\begin{itemize}
    \item{Posicion (latitud y longitud)}
    \item{Velocidad y direcci\'on del movimiento}
    \item{Aceleraci\'on o taza de cambio de la velocidad}
    \item{Nivel de combustible y taza de consumo del combustible}
    \item{Kilometraje actual}
\end{itemize}
Estos parametros son capturados por una computadora que procede a enviarlos a traves de la red celular
mediante internet movil. Para economizar costos, esta computadora solamente esta equipada con
128 bits de memoria o 2 $\mathbf{unsigned\ long}$ a los cuales llamaremos ``memoria1'' y ``memoria2''.

\section*{Tarea \#1 (30\%)}

Indique como codificara todos los parametros anteriores en los dos $\mathbf{unsigned\ long}$. Apoyese
del ejemplo de los colores para indicar que datos van a haber en cada uno de los bits de cada uno de
los 2 $\mathbf{unsigned\ long}$. Indique cuantos bits utilizara para cada uno de los parametros. Puede
hacer aproximaciones para reducir el espacio, por ejemplo, la precision del kilometraje solo es relevante
al ``cien'' m\'as cercano, por lo cual 3271 kilometros podria compactarse a 32. Indique que suposiciones
esta utilizando para dichas compactaciones. En caso que dos $\mathbf{unsigned\ long}$ no sean suficientes,
puede optar ignorar algunos parametros. De igual forma, si solamente un $\mathbf{usigned\ long}$ es
suficiente, puede ignorar el segundo. Se le otorgaran puntos extras por presentar soluciones que usen
solo un $\mathbf{unsigned\ long}$.

\section*{Tarea \#2 (30\%)}

Defina la funci\'on ``codificarMemoriaUno''. Esta funci\'on debe aceptar los parametros necesarios
que provendrian de la \emph{unidad de monitoreo y gps} y produce un $\mathbf{unsigned\ long}$ con
los parametros codificados correspondiente a la ``memoria1''. Usted solamente debe definir la funci\'on,
no se preocupe sobre como esta funci\'on sera llamada. Utilize un parametro diferente para la funci\'on
por cada uno de los parametros de la \emph{unidad de monitoreo y gps}.

\section*{Tarea \#3 (30\%)}

Similar a la \emph{Tarea \#2}, defina la funci\'on ``codificarMemoria2''. Si ud. solamente necesita
la ``memoria1'' para codificar todos los parametros, puede dejar esta pregunta sin responder y
recibira el credito de igual forma.

\section*{Tarea \#4 (10\%)}

Defina una funci\'on \emph{main} donde iniciara la ejecucion de un programa. Esta funci\'on debe
llamar a ``codificarMemoriaUno'' con valores escogidos por usted mismo y almacenar el resultado
en una variable de tipo $\mathbf{unsigned\ long}$ llamada ``memoria1''. Hacer lo mismo con
``memoria2'' si esta utilizando esa memoria tambien.

\end{document}
