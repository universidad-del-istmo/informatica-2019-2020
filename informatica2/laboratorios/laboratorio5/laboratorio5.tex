%%%%%%%%%%%%%%%%%%%%%%%%%%%%%%%%%%%%%%%%%
% Programming/Coding Assignment
% LaTeX Template
%
% This template has been downloaded from:
% http://www.latextemplates.com
%
% Original author:
% Ted Pavlic (http://www.tedpavlic.com)
%
% Note:
% The \lipsum[#] commands throughout this template generate dummy text
% to fill the template out. These commands should all be removed when 
% writing assignment content.
%
% This template uses a Perl script as an example snippet of code, most other
% languages are also usable. Configure them in the "CODE INCLUSION 
% CONFIGURATION" section.
%
%%%%%%%%%%%%%%%%%%%%%%%%%%%%%%%%%%%%%%%%%

%----------------------------------------------------------------------------------------
%	PACKAGES AND OTHER DOCUMENT CONFIGURATIONS
%----------------------------------------------------------------------------------------

\documentclass{article}

\usepackage{fancyhdr} % Required for custom headers
\usepackage{lastpage} % Required to determine the last page for the footer
\usepackage{extramarks} % Required for headers and footers
\usepackage[usenames,dvipsnames]{color} % Required for custom colors
\usepackage{graphicx} % Required to insert images
\usepackage{listings} % Required for insertion of code
\usepackage{courier} % Required for the courier font
\usepackage{multirow}
\usepackage{hyperref}


% Margins
\topmargin=-0.45in
\evensidemargin=0in
\oddsidemargin=0in
\textwidth=6.5in
\textheight=9.0in
\headsep=0.25in

\linespread{1.1} % Line spacing

%----------------------------------------------------------------------------------------
%	CODE INCLUSION CONFIGURATION
%----------------------------------------------------------------------------------------

\definecolor{MyDarkGreen}{rgb}{0.0,0.4,0.0} % This is the color used for comments
\lstloadlanguages{c} % Load Perl syntax for listings, for a list of other languages supported see: ftp://ftp.tex.ac.uk/tex-archive/macros/latex/contrib/listings/listings.pdf
\lstset{language=[sharp]c, % Use Perl in this example
        frame=single, % Single frame around code
        basicstyle=\small\ttfamily, % Use small true type font
        keywordstyle=[1]\color{Blue}\bf, % Perl functions bold and blue
        keywordstyle=[2]\color{Purple}, % Perl function arguments purple
        keywordstyle=[3]\color{Blue}\underbar, % Custom functions underlined and blue
        identifierstyle=, % Nothing special about identifiers                                         
        commentstyle=\usefont{T1}{pcr}{m}{sl}\color{MyDarkGreen}\small, % Comments small dark green courier font
        stringstyle=\color{Purple}, % Strings are purple
        showstringspaces=false, % Don't put marks in string spaces
        tabsize=5, % 5 spaces per tab
        %
        % Put standard Perl functions not included in the default language here
        morekeywords={rand},
        %
        % Put Perl function parameters here
        morekeywords=[2]{on, off, interp},
        %
        % Put user defined functions here
        morekeywords=[3]{test},
       	%
        morecomment=[l][\color{Blue}]{...}, % Line continuation (...) like blue comment
        numbers=left, % Line numbers on left
        firstnumber=1, % Line numbers start with line 1
        numberstyle=\tiny\color{Blue}, % Line numbers are blue and small
        stepnumber=5 % Line numbers go in steps of 5
}

\newcommand{\horrule}[1]{\rule{\linewidth}{#1}}

\newcommand\doubleplus{\ensuremath{\mathbin{+\mkern-10mu+}}}

% Creates a new command to include a perl script, the first parameter is the filename of the script (without .pl), the second parameter is the caption
\newcommand{\perlscript}[2]{
\begin{itemize}
\item[]\lstinputlisting[caption=#2,label=#1]{#1.cs}
\end{itemize}
}

\begin{document}

\begin{tabular}{l l}
\multirow{5}{*}{\includegraphics[width=2cm]{../../recursos/logo.png}} & Universidad del Istmo de Guatemala \\
 & Facultad de Ingenieria \\
 & Ing. en Sistemas \\
 & Informatica II \\
 & Prof. Ernesto Rodriguez - \href{mailto:erodriguez@unis.edu.gt}{erodriguez@unis.edu.gt} \\
\end{tabular}
\\\\\\

\begin{center}
        \horrule{0.5pt}
        \huge{Laboratorio \#4} \\
        \large{Fecha de entrega: 5 de Marzo, 2020 - 11:59pm} \\
        \horrule{1pt}
\end{center}

\emph{Instrucciones: Resolver cada uno de los ejercicios siguiendo sus respectivas
instrucciones. El trabajo debe ser entregado a traves de Github, en su repositorio del curso, colocado en una carpeta llamada "Laboratorio \#3".
Este laboratorio debe ser elaborado en parejas.}

\section*{Tarea \#1 (10\%)}
Defina la funci\'on ``$\mathbf{int*}\ copiar(\mathbf{int*}\ valores,\ \mathbf{int}\ cantidad)$''. Esta funci\'on
acepta un arreglo de numeros y el tama\~no del arreglos. La funci\'on debe retornar un arreglo nuevo, creado
en el \emph{heap} del programa, que sea una copia exacta del arreglo que se le dio como parametro.

\section*{Tarea \#2 (50\%)}

Defina la funci\'on ``$\mathbf{int*}\ primos(\mathbf{const\ int}\ maximo)$''. Esta funci\'on debe retornar un arreglo
con todos los numeros primos menores a ``$maximo$''. Asegurese que su funci\'on:
\begin{itemize}
        \item{No reserve m\'as memoria de la necesaria.}
        \item{Limpie utilizando $\mathbf{delete}$ la memoria que ya no se necesita.}
        \item{El puntero retornado como valor apunta a memoria ubicada en el \emph{heap}}
\end{itemize}

\section*{Tarea \#3 (40\%)}

Defina la funci\'on ``$\mathbf{int**}\ vectoresCercanos(\mathbf{int**}\ vectores,\ \mathbf{const\ int}\ cantidad,\ \mathbf{const\ int}\ distancia)$''.
Esta funci\'on acepta como parametro un arreglo de vectores de dos dimensiones (con coordenadas ``X'' y ``Y''), el
numero de vectores y una distancia. La funci\'on debe retornar un arreglo con todos los vectores que se encuentren
a menos de ``distancia'' unidades (utilizando la distancia euclideana) del vector con coordenadas $\langle 0,0 \rangle$.
Asegurese que los vectores retornados son punteros \emph{nuevos}, en otras palabras, debe reservar nueva memoria
tanto para el arreglo retornado como para cada uno de los vectores dentro del mismo.

\end{document}
