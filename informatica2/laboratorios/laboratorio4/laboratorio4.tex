%%%%%%%%%%%%%%%%%%%%%%%%%%%%%%%%%%%%%%%%%
% Programming/Coding Assignment
% LaTeX Template
%
% This template has been downloaded from:
% http://www.latextemplates.com
%
% Original author:
% Ted Pavlic (http://www.tedpavlic.com)
%
% Note:
% The \lipsum[#] commands throughout this template generate dummy text
% to fill the template out. These commands should all be removed when 
% writing assignment content.
%
% This template uses a Perl script as an example snippet of code, most other
% languages are also usable. Configure them in the "CODE INCLUSION 
% CONFIGURATION" section.
%
%%%%%%%%%%%%%%%%%%%%%%%%%%%%%%%%%%%%%%%%%

%----------------------------------------------------------------------------------------
%	PACKAGES AND OTHER DOCUMENT CONFIGURATIONS
%----------------------------------------------------------------------------------------

\documentclass{article}

\usepackage{fancyhdr} % Required for custom headers
\usepackage{lastpage} % Required to determine the last page for the footer
\usepackage{extramarks} % Required for headers and footers
\usepackage[usenames,dvipsnames]{color} % Required for custom colors
\usepackage{graphicx} % Required to insert images
\usepackage{listings} % Required for insertion of code
\usepackage{courier} % Required for the courier font
\usepackage{multirow}
\usepackage{hyperref}


% Margins
\topmargin=-0.45in
\evensidemargin=0in
\oddsidemargin=0in
\textwidth=6.5in
\textheight=9.0in
\headsep=0.25in

\linespread{1.1} % Line spacing

%----------------------------------------------------------------------------------------
%	CODE INCLUSION CONFIGURATION
%----------------------------------------------------------------------------------------

\definecolor{MyDarkGreen}{rgb}{0.0,0.4,0.0} % This is the color used for comments
\lstloadlanguages{c} % Load Perl syntax for listings, for a list of other languages supported see: ftp://ftp.tex.ac.uk/tex-archive/macros/latex/contrib/listings/listings.pdf
\lstset{language=[sharp]c, % Use Perl in this example
        frame=single, % Single frame around code
        basicstyle=\small\ttfamily, % Use small true type font
        keywordstyle=[1]\color{Blue}\bf, % Perl functions bold and blue
        keywordstyle=[2]\color{Purple}, % Perl function arguments purple
        keywordstyle=[3]\color{Blue}\underbar, % Custom functions underlined and blue
        identifierstyle=, % Nothing special about identifiers                                         
        commentstyle=\usefont{T1}{pcr}{m}{sl}\color{MyDarkGreen}\small, % Comments small dark green courier font
        stringstyle=\color{Purple}, % Strings are purple
        showstringspaces=false, % Don't put marks in string spaces
        tabsize=5, % 5 spaces per tab
        %
        % Put standard Perl functions not included in the default language here
        morekeywords={rand},
        %
        % Put Perl function parameters here
        morekeywords=[2]{on, off, interp},
        %
        % Put user defined functions here
        morekeywords=[3]{test},
       	%
        morecomment=[l][\color{Blue}]{...}, % Line continuation (...) like blue comment
        numbers=left, % Line numbers on left
        firstnumber=1, % Line numbers start with line 1
        numberstyle=\tiny\color{Blue}, % Line numbers are blue and small
        stepnumber=5 % Line numbers go in steps of 5
}

\newcommand{\horrule}[1]{\rule{\linewidth}{#1}}

\newcommand\doubleplus{\ensuremath{\mathbin{+\mkern-10mu+}}}

% Creates a new command to include a perl script, the first parameter is the filename of the script (without .pl), the second parameter is the caption
\newcommand{\perlscript}[2]{
\begin{itemize}
\item[]\lstinputlisting[caption=#2,label=#1]{#1.cs}
\end{itemize}
}

\begin{document}

\begin{tabular}{l l}
\multirow{5}{*}{\includegraphics[width=2cm]{../../recursos/logo.png}} & Universidad del Istmo de Guatemala \\
 & Facultad de Ingenieria \\
 & Ing. en Sistemas \\
 & Informatica II \\
 & Prof. Ernesto Rodriguez - \href{mailto:erodriguez@unis.edu.gt}{erodriguez@unis.edu.gt} \\
\end{tabular}
\\\\\\

\begin{center}
        \horrule{0.5pt}
        \huge{Laboratorio \#4} \\
        \large{Fecha de entrega: 27 de Febrero, 2019 - 11:59pm} \\
        \horrule{1pt}
\end{center}

\emph{Instrucciones: Resolver cada uno de los ejercicios siguiendo sus respectivas
instrucciones. El trabajo debe ser entregado a traves de Github, en su repositorio del curso, colocado en una carpeta llamada "Laboratorio \#3".
Este laboratorio debe ser elaborado en parejas.}

\section*{Tarea \#1 (25\%)}

Defina la funci\'on $\mathbf{bool}\ divisionSegura(\mathbf{int}\ numerador,\ \mathbf{int}\ denomindador,\ \mathbf{int*}\ respuesta)$.
Esta funci\'on debe calcular la division entre dos numeros validando que no se este utilizando 0 en el denominador.
En caso que la division sea possible, esta funci\'on debe hacer la division y guardarla en el puntero ``respuesta'' y
retornar $\mathtt{true}$ de lo contrario, debe retornar $\mathtt{false}$.

\section*{Tarea \#2 (25\%)}

Defina una funci\'on llamada ``$\mathbf{bool}\ sumaMayor(\mathbf{int[]}\ valores,\ \mathbf{int}\ cantidad,\ \mathbf{int**}\ respuesta)$''. Esta funci\'on debe aceptar un arreglo de numeros enteros
y buscar los dos numeros que generen la mayor suma en este arreglo. Estos dos numeros se deben almacenar en el
puntero que se recibe como segundo parametro utilizando el indice 0 para el primer valor y el indice 1 para el segundo valor.
Esta funci\'on retorna $\mathtt{true}$ cuando el arreglo recibido tiene al menos dos elementos, de lo contrario,
no se hace ninguna operaci\'on y se retorna false.

\section*{Tarea \#3 (40\%)}

Defina la funci\'on $\mathbf{void}\ fibonacciN(\mathbf{const int}\ n,\mathbf{int*}\ valores)$. Esta funci\'o
acepta un numero ``$n$'', produce los primeros $n$ numeros de fibonacci y los almacena en el arreglo
``valores'' que se le ha dado como parametro.

\section*{Tarea \#4 (10\%)}

Optimize la func\'on ``$fibonacciN$'' de la tarea 3 de tal forma que pueda calcular los numeros de
fibonacci eficientemetne. Puede probar que su implementaci\'on funciona utilizando un numero
grande como parametro. Por ejemplo, calcular los primeros 10,000 numeros de fibonacci.

\end{document}
