%%%%%%%%%%%%%%%%%%%%%%%%%%%%%%%%%%%%%%%%%
% Programming/Coding Assignment
% LaTeX Template
%
% This template has been downloaded from:
% http://www.latextemplates.com
%
% Original author:
% Ted Pavlic (http://www.tedpavlic.com)
%
% Note:
% The \lipsum[#] commands throughout this template generate dummy text
% to fill the template out. These commands should all be removed when 
% writing assignment content.
%
% This template uses a Perl script as an example snippet of code, most other
% languages are also usable. Configure them in the "CODE INCLUSION 
% CONFIGURATION" section.
%
%%%%%%%%%%%%%%%%%%%%%%%%%%%%%%%%%%%%%%%%%

%----------------------------------------------------------------------------------------
%	PACKAGES AND OTHER DOCUMENT CONFIGURATIONS
%----------------------------------------------------------------------------------------

\documentclass{article} 

\usepackage{fancyhdr} % Required for custom headers
\usepackage{lastpage} % Required to determine the last page for the footer
\usepackage{extramarks} % Required for headers and footers
\usepackage[usenames,dvipsnames]{color} % Required for custom colors
\usepackage{graphicx} % Required to insert images
\usepackage{listings} % Required for insertion of code
\usepackage{courier} % Required for the courier font
\usepackage{multirow}
\usepackage{hyperref}
\usepackage{amsmath}
\usepackage{amssymb}


% Margins
\topmargin=-0.45in
\evensidemargin=0in
\oddsidemargin=0in
\textwidth=6.5in
\textheight=9.0in
\headsep=0.25in

\linespread{1.1} % Line spacing

%----------------------------------------------------------------------------------------
%	CODE INCLUSION CONFIGURATION
%----------------------------------------------------------------------------------------

\definecolor{MyDarkGreen}{rgb}{0.0,0.4,0.0} % This is the color used for comments
\lstloadlanguages{c} % Load Perl syntax for listings, for a list of other languages supported see: ftp://ftp.tex.ac.uk/tex-archive/macros/latex/contrib/listings/listings.pdf
\lstset{language=[sharp]c, % Use Perl in this example
        frame=single, % Single frame around code
        basicstyle=\small\ttfamily, % Use small true type font
        keywordstyle=[1]\color{Blue}\bf, % Perl functions bold and blue
        keywordstyle=[2]\color{Purple}, % Perl function arguments purple
        keywordstyle=[3]\color{Blue}\underbar, % Custom functions underlined and blue
        identifierstyle=, % Nothing special about identifiers                                         
        commentstyle=\usefont{T1}{pcr}{m}{sl}\color{MyDarkGreen}\small, % Comments small dark green courier font
        stringstyle=\color{Purple}, % Strings are purple
        showstringspaces=false, % Don't put marks in string spaces
        tabsize=5, % 5 spaces per tab
        %
        % Put standard Perl functions not included in the default language here
        morekeywords={rand},
        %
        % Put Perl function parameters here
        morekeywords=[2]{on, off, interp},
        %
        % Put user defined functions here
        morekeywords=[3]{test},
       	%
        morecomment=[l][\color{Blue}]{...}, % Line continuation (...) like blue comment
        numbers=left, % Line numbers on left
        firstnumber=1, % Line numbers start with line 1
        numberstyle=\tiny\color{Blue}, % Line numbers are blue and small
        stepnumber=5 % Line numbers go in steps of 5
}

\newcommand{\horrule}[1]{\rule{\linewidth}{#1}}

% Creates a new command to include a perl script, the first parameter is the filename of the script (without .pl), the second parameter is the caption
\newcommand{\perlscript}[2]{
\begin{itemize}
\item[]\lstinputlisting[caption=#2,label=#1]{#1.cs}
\end{itemize}
}

\begin{document}

\begin{tabular}{l l}
\multirow{5}{*}{\includegraphics[width=2cm]{../recursos/logo.png}} & Universidad del Istmo de Guatemala \\
 & Facultad de Ingenieria \\
 & Ing. en Sistemas \\
 & Informatica II \\
 & Prof. Ernesto Rodriguez - \href{mailto:erodriguez@unis.edu.gt}{erodriguez@unis.edu.gt} \\
\end{tabular}
\\\\\\

\begin{center}
        \horrule{0.5pt}
        \huge{Proyecto Final: Supermercado} \\
        \large{Fecha de entrega: 11 de Mayo, 2020 - 11:59pm} \\
        \horrule{1pt}
\end{center}

\section*{Descripci\'on}

A continuaci\'on se presenta el proyecto final el cual tiene como objetivo que el estudiante
ponga en practica sus habilidades de programar y resoluci\'on de problemas para un problema
del mundo real.
\\\\
El contexto de este proyecto es que usted ha sido contratado por un supermercado que quiere
optimizar la forma en que ubica sus productos dentro de una tienda con el proposito de facilitarle
las compras a sus clientes. Para ello, debera escribir un programa que busque entre los articulos
que han comprado los clientes del supermercado, cuales son los articulos que se compran frecuentemente
en conjunto.
\\\\
Su programa debe leer linea por linea listas de articulos que han comprado los clientes del supermercado.
Cada articulo se representara por un numero entero (para facilitar el trabajo) en vez de utilizar el
nombre del articulo o cualquier otra forma de representarlo. Cada linea de texto representa una lista
de articulos que compro un cliente separada por coma. Por ejemplo, su programa podria recibir las
siguientes lineas de texto como entraea:
\begin{itemize}
        \item{42,50,11,1,61}
        \item{11,61,88,33,50,2}
        \item{1,7,50,30,22,33,42}
\end{itemize}
En este ejemplo se puede apreciar que los articulos 42 y 50 fueron comprados en conjunto por
2 clientes.
\\\\
Luego de haber leido todas las listas de articulos (se utilizara una linea vacia para indicar el final)
su programa debera imprimir una lista que:
\begin{itemize}
        \item{Tiene todas las parejas de articulos frecuentemente comprados en conjunto.}
        \item{Muestra el numero de veces que estos articulos se han comprado en conjunto.}
        \item{La lista esta ordenada de grupos m\'as frecuentes a menos frecuentes.}
\end{itemize}

Su trabajo debe ser entregado mediante un repositorio en git. Asegurese que su codigo
sea ordenado y este bien comentado para facilitar su revision. El trabajo puede hacerse
de forma individual o en parejas. La nota, sin embargo, sera otorgada de forma individual.
Ambos recibiran el mismo punteo por el codigo, pero esa nota sera ponderada por el conocimiento
individual que tenga cada miembro del codigo el cual sera evaluado mediante una evaluaci\'on
oral del codigo.

\section*{Extra: (5 puntos netos)}

Extienda el programa para que no solamente busque las parejas de articulos comprados frecuentemente,
sino que haga un conteo de los grupos de articulos de tama\~no arbitrario que se compran con
frecuencia. Los resultados deben ser odrdenados primero por el tama\~no del grupo de articulos y
luego por la cantidad de veces que aparece cada uno de los grupos.

\section*{Extras: (5 puntos netos)}

Se otorgaran puntos extras por entregar un programa con mejoras adicionales. Estas pueden
ser:
\begin{itemize}
        \item{Interfaz grafica}
        \item{Pruebas unitarias (cpptest o googletest)}
        \item{Utilizaci\'on de una base de datos para realizar el trabajo}
        \item{Utilizar gnuplot u otra herramienta para generar un grafico con los resultados del programa.}
\end{itemize}

\end{document}